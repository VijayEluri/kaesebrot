
\chapter{Ergebnis}
\label{chapter_ergebnis}

Im Zuge der vorliegenden Arbeit wurde eine Domain Specific Language (DSL) entwickelt und in ein existierendes Studiensystem integriert.

Die entwickelte DSL, die Form Expression Language (FXL), wurde als eigenständige Java-Bibliothek entworfen. Als solche ist diese unabhängig vom konkreten Einsatzsystem. Über die entwickelten Schnittstellen ist die Bibliothek somit in beliebige Java-Anwenundgen integrierbar.

Beim Entwurf der FXL wurde auf eine einfache Syntax, die sich an bekannten Anwendungen wie Tabellenkalkulationssoftware oder Taschenrechner orientiert, Wert gelegt. Die Sprache definiert einzelne Statements, die durch logische Operatoren, arithmetische Operatoren und Vergleichsoperatoren, sowie Klammerungen, definiert werden. Ferner enthält die Syntax auch Variablen, die als Referenzen auf Werte  in die Statements eingebunden werden können. Über eine Schnittstelle kann eine externe Implementierung einer Symboltabelle eingebunden werden, die Variablennamen in konkrete Werte auflöst. Zusätzlich können in Java entwickelte, wohldefinierte Funktionen in die Sprache integriert werden. Die FXL unterstützt vier Datentypen: Ganzzahlen, Gleitkommazahlen, Datumswerte und boolesche Werte.

Um dem Benutzer Feedback über die eingegebenen Statements geben zu können, existiert eine detaillierte Fehlerbehandlung über eine Hierarchie von eigens definierten Exceptions. Diese Exceptions bilden Fehler in den Phasen syntaktische Analyse, semantische Analyse, sowie Laufzeit ab. Jede dieser Fehlerkategorien definiert wiederum detaillierte Fehlercodes und Parameter, um die Ursache eines Fehlers einfach nachvollziehbar darzustellen.

Bei der Entwicklung der FXL wurde auf einen testgetriebenen Entwicklungsablauf achtgegeben. Für jedes neue Feature (diverse Operatoren, Variablen, Funktionen, Fehler etc.) wurden vor der Implementierung Testfälle definiert, die das korrekte Verhalten der Sprache abbilden. Dadurch konnte eine effiziente Entwicklung und ein testbares Design erreicht werden.

Bei der Integration der FXL in eine bestehende Anwendung sind unterschiedliche Aspekte zu beachten. Als Integrationsbeispiel wird die Einbettung der DSL in die medizinische Studiensoftware SPICS vorgenommen.

Zuerst muss der Workflow des Zielsystems analysiert werden. Dies ist notwendig, um die Anknüpfungspunkte für die Eingabe der Ausdrücke sowie die Ausführung des Interpreters zu ermitteln. Der nächste Schritt sind die Änderungen im Datenmodell, um die Statements im System zu persistieren.

Bei der Integration in eine Webanwendung wie SPICS, sind Gedanken über den Ort und die Zeit der Ausführung unerlässlich. Durch die Trennung in Client und Server, bieten sich mehrere Möglichkeiten mit unterschiedlichen Stärken und Schwächen. Für die Integration in diese Arbeit wurde eine serverseitige Lösung gewählt. Die Statements werden allerdings in Echtzeit, bereits bei der Eingabe der Daten, per Ajax-Calls an den Server übertragen und dort ausgewertet.

Da Variablen in den FXL-Statements im Studiensystem auf andere Formularfelder im selben Formular zeigen, muss gewährleistet werden, dass sich diese in abhängigen Formularfeldern nicht gegenseitig referenzieren. Dazu wird ein Baum von Abhängigkeiten der Formularfelder gebildet, der auf Zyklenfreiheit untersucht wird. Dieser Baum gibt auch die Ausführungsreihenfolge der Statements vor. Vor der Auswertung eines Statements, müssen zuerst die von diesem referenzierten Felder ausgewertet werden.

Zuletzt müssen auch die Fehler, die bei Eingabe der Statements bzw. bei deren Ausführung auftreten können, behandelt werden. Da die semantische Analyse und die Ausführung am Server vorgenommen wird, können die in der FXL definierten Fehler abgefangen, und an den Client zurückgegeben werden. Dabei erfolgt über die Fehlercodes eine Umwandlung der Fehler in applikationsabhängige, internationalisierte Fehlermeldungen.



\chapter{Ausblick}
\label{chapter_ausblick}

Im Verlauf der Arbeit haben sich einige Gedanken herauskristallisiert, die sich für weitere Arbeiten, über den Umfang dieser Diplomarbeit hinaus, anbieten. Diese  Themen werden im Folgenden kurz beschrieben. 

In der Implementierung wurde die Integration der Form Expression Language in SPICS nur für drei verschiedene Formelemente vorgenommen, die ausreichen, um die vier unterstützten Datentypen zu repräsentieren: Textfelder für die numerischen Typen, sowie Checkboxen für boolesche Werte und Datepicker für Datumswerte. Für die Verwendung von weiteren Formelementen wie Listen, Drop-Downs und Radiobuttons können Überlegungen angestellt werden, wie Werte dieser Elemente mit der FXL verarbeitet werden können.

Referenzierungen in den Formeln der FXL lassen sich momentan nur auf Formularfelder innerhalb eines Formulars vornehmen. Da für einen Patienten in SPICS unterschiedliche Formulare angelegt werden können, kann die Referenzierung von Feldern aus anderen Formularen eine weitere Anforderung darstellen. Dafür kann die Implementierung der Variable Provider dahingehend geändert werden, dass ein Lookup von Variablen auch Werte bereits persistierter Formulare berücksichtigt. 

Eine weitere Form von Abhängigkeiten zwischen Formularfeldern könnten eingabeabhängige Berechnungen darstellen. Damit sind Abhängigkeiten zwischen Feldern gemeint, die Werte je nach vorhandener Eingabe berechnen. Im Beispiel des Body Mass Index (BMI) Rechners, der aus Körpergröße und Körpergewicht den BMI berechnet, könnte eine solche eingabeabhängige Berechnungwie folgt aussehen. Aus zwei beliebigen Feldern wird der dritte Wert automatisch berechnet. Werden also zuerst das Gewicht und der BMI eingegeben, wird automatisch die Körpergröße ermittelt. Inwieweit das durch Umformen von Statements der FXL möglich ist, ist zu evaluieren.

Auch die Form Expression Language selbst lässt Raum für Erweiterungen in mancherlei Hinsicht. Für eine formularübergreifende Referenzierung könnte etwa die Syntax für Variablennamen modifiziert werden. Eine weitere mögliche Erweiterung stellt die Einführung eines Datentyps für Zeichenketten, mit entsprechen Auswirkungen auf die Syntax der Sprache sowie die Semantik der Operatoren, dar. Eine zusätzliche Möglichkeit stellt die Definition weiterer Operatoren dar. In der Implementierung dieser Arbeit kann der ganzzahlige Rest einer Division (Modulo) nur durch eine eigene Funktion berechnet werden. Für diese Berechnung könnte, wie in anderen Sprachen üblich, der Operater \texttt{\%} eingeführt werden. Auch eigene logische Operatoren für exklusives Oder und Implikation sind denkbar. Diese logischen Operatoren müssen momentan durch die vorhandenen Grundoperatoren ausgedrückt werden.

Ein weiteres Thema sind die nichtfunktionalen Eigenschaften der Integration im Studiensystem. Vor allem in Hinsicht Usability kann die Implementierung einer Evaluierung unterzogen werden. Die Akzeptanz beim Endanwender kann durch User-Tests bestätigt oder widerlegt werden. Auch die Eingabe der Statements lässt Raum für Verbesserungen im User Interface. Eine Möglichkeit wäre beispielsweise Syntax Highlighting oder Code Completion anzubieten, um die Eingabe angenehmer zu gestalten und Fehler zu vermeiden. Auch in Hinblick auf die Performance der Sprache, sowie die Integration in das Studiensystem, können Untersuchungen vorgenommen werden. Vor allem die Möglichkeiten einer teilweisen clientseitigen Berechnung, wie in Abschnitt \ref{integration_ausfuehrungsort} dargelegt, können Thema einer weiteren Untersuchung sein.

\chapter{Zusammenfassung}
\label{chapter_zusammenfassung}

Das Thema dieser Arbeit entstand aus der Anforderung, Formulare in einer Webanwendung semantisch dahingehend anzureichern, dass flexible Berechnungen und Validierungen formelementübergreifend vorgenommen werden können. Im Rahmen der Arbeit wurde eine einfache Ausdruckssprache entwickelt und in ein medizinisches Studiensystem integriert.

Der erste Teil der Arbeit beschäftigt sich mit den technischen und theoretischen Grundlagen, die zum Verständnis des Textes notwendig sind. Zunächst wurde der Begriff der Domain Specific Language (DSL) diskutiert, eine Einteilung vorgenommen und Beispiele für bekannte DSLs vorgestellt. Danach wurden die theoretischen Grundlagen erarbeitet und eine Übersicht über Werkzeuge zum Entwurf von Sprachen geboten. Den Abschluss des ersten Teils bildet eine Übersicht über wissenschaftliche Arbeiten im Bereich der Modellierung von Webformularen.

Im zweiten Teil wird der Entwicklungsprozess der Form Expression Language (FXL), der sich in die Phasen Entscheidung, Analyse, Entwurf, Implementierung und Test gliedert, beschrieben. Zuerst wurde eine Entscheidung für eine eigene externe DSL getroffen und begründet. Danach wurden die Anforderungen an die Sprache analysiert und das Design der Sprache in Hinsicht auf die notwendigen Features wie Datentypen, Operatoren, Funktionen und Fehlerbehandlung, sowie Syntax und Semantik entwickelt. Danach wurden die Implementierungsdetails der FXL dargelegt. Den Abschluss des Teils bildet eine Abhandlung über das Testen der erstellten Software.

Der dritte Teil behandelt die Integration der Sprache in ein existierendes medizinisches Studiensystem. Zuerst wurde das Studiensystem SPICS kurz vorgestellt und die Architektur sowie die verwendeten Technologien beschrieben. Danach wurden die verschiedenen Aspekte der Integration der Sprache in SPICS veranschaulicht.

Der vierte Teil gibt einen Ausblick auf weitere Themen und Möglichkeiten, die sich im Zuge der Arbeit aufgetan haben und bietet eine zusammenfassende Übersicht über die Themen der Arbeit.

Ein Schwerpunkt der Arbeit war die praktische Implementierung der FXL, sowie die Integration in das medizinische Studiensystem. Die Arbeitsschritte entsprechen im Grunde den in der Arbeit dargestellten Schritten, mit dem Unterschied, dass die praktische Umsetzung natürlich ein iterativer Prozess war. Zuerst wurden die Anforderungen und das Zielsystem analysiert und eine Übersicht über das wissenschaftliche Feld erstellt. Danach wurden diverse Tools und Möglichkeiten zur Umsetzung der Sprache in Hinsicht auf die Anforderungen evaluiert. Im nächsten Schritt erfolgte die testgetriebene Implementierung der FXL, in der die Anforderungen iterativ überarbeitet und immer detaillierter spezifiziert wurden. Eine besondere Herausforderung war die Integration in das vorhandene Studiensystem, da auf die Besonderheiten des Systems, wie die Technologien oder das Datenmodell, eingegangen werden musste. Das Ergebnis der praktischen Arbeit ist eine funktionierende Implementierung gemäß der Problemstellung der schriftlichen Arbeit.



