\chapter{Einleitung}

Die voriegende Arbeit bewegt sich im Umfeld einer medizinischen Dokumentationssoftware, die zur Eingabe und
Verwaltung von medizinischen Daten verwendet wird. Die Software ist eine Webanwendung, die
Krankenhäusern, Ärzten und anderen Spezialisten den gemeinsamen Zugriff auf Daten für medizinische
Studien ermöglicht. Dabei ist jedoch die Anonymität der Studienteilnehmer gesichert, da medizinische Daten
nur pseudonymisiert und getrennt von Patientendaten abgelegt werden.

Die Software bietet die Möglichkeit, für verschiedene medizinische Fachbereiche bzw. Anforderungen
individuelle Eingabemasken zu erstellen. Dafür können aus unterschiedlichen Formularfeld-Typen flexibel
benutzerdefinierte Formulare erstellt werden. Manche Felder unterliegen dabei Anforderungen, was den Typ
der eingegebenen Daten bzw. den Datenbereich betrifft. In dieser Arbeit soll eine Domain Specific Language (DSL) entwickelt werden, 
welche eine erweiterte Modellierung von Abhängigkeiten und Validierungen in den Eingabemasken ermöglicht.


\section{Motivation}

Die Notwendigkeit für eine generische Lösung zur Berechnung und Eva\-lu\-ie\-rung von Formularelementen
ist begründet durch das Bedürfnis der Benutzer nach Formularen, die automatisch Berechnungen über
mehrere Formulare durchführen. Solche Formulare wurden bisher als eigenes Plugin realisiert, das statt dem Formular 
angezeigt wird. Das Plugin ist somit genau auf den konkreten Anwendungsfall zugeschnitten. Der Nachteil dieser
Lösung ist allerdings, dass für jede Änderung der Sourcecode des Plugins bearbeitet, und die Anwendung 
neu deployed werden muss. Ein weiterer Nachteil ist, das für jeden weiteren Anwendungsfall mit anderen Berechnungen
ein weiteres Plugin notwendig ist.

Daraus ist die Anforderung nach einer allgemeineren und flexibleren Lösung zur individuellen Modellierung von 
Berechnungen in Formularen entstanden.


\section{Problemstellung}

Um eine flexible Modellierung von Berechnungen in Formularen zu erreichen, ist es notwendig, einzelne Felder
automatisch durch arithmetische Operationen mit Werten aus anderen Feldern zu berechnen, wie dies aus diversen
Tabellenkalkulationsprogrammen bekannt ist. Dazu sind zwei verschiedene Arten von Statements notwendig:
Formeln, die einem Feld einen Wert zuweisen und Bedingungen, die den Wert eines Formularfeldes
über\-prü\-fen. Die Eingabe der Statements soll direkt bei der Erstellung eines Formulars in der Webanwendung
geschehen. So ist es beispielsweise zielführend, ein Feld für den Body Mass Index automatisch aus
Körpergröße und Gewicht berechnen zu lassen, anstatt es manuell auszufüllen und bei jeder Änderung der
Daten neu einzugeben.

Ein weiterer wichtiger Teil der Problemstellung ist die Einbettung der Berechnungen und Constraints in das
bestehende Softwaresystem. Da dieses System bereits in Verwendung ist, ist es eine besondere
Herausforderung, die in der Arbeit vorgenommene Implementierung der DSL in das Studiensystem zu integrieren,
ohne die bestehende Architektur oder das Datenmodell im Hinblick auf eine notwendige Migration
maßgeblich zu verändern.


\section{Zielsetzung}

Das Ziel der Diplomarbeit ist, eine Domain Specific Language (DSL) zu entwickeln. Dazu ist es notwendig, zu definieren, was eine DSL
ist und was in diesem konkreten Anwendungsfall die Domäne darstellt. Die DSL soll es ermöglichen,
Beziehungen zwischen Formularfeldern herzustellen. Dabei ist darauf zu achten, dass die Syntax einfach
und konsistent ist. Die DSL soll in ihrer Syntax der Schreibweise von bekannten Systemen (Taschenrechner,
Tabellenkalkulation, …) ähnlich sein, so dass diese für erfahrene Computernutzer gut erlernbar und
anwendbar ist.

Der Kern der Arbeit ist der Entwurf und die Entwicklung einer do\-mä\-nen\-spe\-zi\-fi\-schen Sprache und deren
Integration in das bestehende Studiensystem. Der Fokus liegt dabei auf der Funktionalität der
domänenspezifischen Sprache. Nicht Teil der Arbeit sind formularübergreifende Referenzierungen, welche
jedoch in Konzeption und Design berücksichtigt werden. 

Ein Schwerpunkt der Arbeit ist die Integration der DSL in die bestehende Webanwendung. Dabei wird darauf
geachtet, dass die Referenzierung der verschiedenen Formularfelder für Formeln, die Werte aus
mehreren Feldern benötigen, für den Endbenutzer einfach und nachvollziehbar erfolgt. Weiters wird eine
Möglichkeit geschaffen, die Statements der DSL direkt in der Webanwendung zu den
entsprechenden Formularfeldern einzugeben. Schließlich wird die Ausführung der DSL in die Applikation
integriert, das heißt, wenn Daten eingegeben werden, müssen die angegeben Formeln berechnet, bzw.
Bedingungen überprüft werden. Inwieweit die Ausführung client- oder serverseitig geschieht, wird auch im
Hinblick auf Usability und Antwortzeiten zu betrachtet.




\section{Struktur}

Die vorliegende Arbeit ist in vier Teile gegliedert.

Der erste Teil, \emph{\nameref{part_grundlagen}}, beschäftigt sich mit der Aufbereitung der für das Verständnis der Arbeit notwendigen fachlichen Themen. Zuerst wird versucht, eine passende Definition des Begriffs `Domain Specific Languages' zu finden (Kapitel \ref{chapter_dsl} \nameref{chapter_dsl}). Kapitel \ref{chapter_theoretische_grundlagen} gibt eine Übersicht über die theoretischen Grundlagen, die zum Verständnis der Arbeit beitragen. In Kapitel \ref{chapter_tools} werden verschiedene Werkzeuge für die Entwicklung vor\-ge\-stel\-lt und insbesondere auf den Parsergenerator ANTLR eingegangen. Ab\-schlie\-ßend wird in Kapitel \ref{related_work} das wissenschaftliche Umfeld aufbereitet.

Im zweiten Teil der Arbeit, \emph{\nameref{part_entwicklung}}, werden die Entwicklungsschritte der aus dieser Arbeit resultierenden DSL beschrieben. Zuerst wird in Kapitel \ref{chapter_entscheidung} die Entscheidung für die Entwicklung einer eigenen Sprache begründet und die Alternativen besprochen. Danach werden in der Analyse die Anforderungen an die Sprache erhoben (Kapitel \ref{chapter_analyse}). In Kapitel \ref{chapter_entwurf} wird die DSL beschrieben. Datentypen, semantische Regeln und weitere Features werden genau spezifiziert. Kapitel \ref{chapter_implementierung} beschreibt die Implementierungsdetails der entwickelten Sprache. Kapitel \ref{chapter_test} widmet sich dem Testing in Hinsicht auf die Implementierung.

Der dritte Teil widmet sich der \emph{\nameref{part_integration}}. In Kapitel \ref{chapter_systembeschreibung} wird das medizinische Studiensystem beschrieben, in das die DSL integriert wird. Dazu wird der Workflow erklärt, um dem Leser den Nutzen der DSL im Kontext des Zielsystems zu erläutern. Zusätzlich wird die Architektur sowie der verwendete Technologie-Stack des Zielsystems beschrieben.

Der vierte Teil, \emph{\nameref{part_ergebnis}}, widmet sich in Kapitel \ref{chapter_ausblick} einem Ausblick auf weitere Themen, die über den Umfang dieser Arbeit hinausgehen und Raum für weitere Arbeiten geben. Zuletzt wird die Arbeit in Kapitel \ref{chapter_zusammenfassung} zusammengefasst.




